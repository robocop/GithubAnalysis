\documentclass[10pt, compress]{beamer}

\usepackage[utf8]{inputenc}
\usepackage[english]{babel}
\usepackage[T1]{fontenc}

\usepackage{booktabs}
\usepackage{tikz}
\usepackage[ruled,vlined,english]{algorithm2e}
\usetikzlibrary{arrows,automata,shapes,positioning,calc}

\usepackage{float} % placement des figures
\usepackage{amssymb} % symboles mathématiques
\usepackage{graphicx} % affichage d'images
\usepackage{url} % inclure des urls

\title{GitHub activity}
\subtitle{}
\date{January 23, 2015}
\institute{École Normale Supérieure de Lyon}
\author{Quentin Cormier, Tom Cornebize, Yassine Hamoudi}

\begin{document}

\maketitle


\begin{frame} % Events on 48h the 1 and 2 January 2015
    % From http://www.texample.net/tikz/examples/pie-chart-color/
    \def\angle{0}
    \def\radius{3}
    \def\cyclelist{{"orange","blue","red","green"}}
    \newcount\cyclecount \cyclecount=-1
    \newcount\ind \ind=-1
    \begin{figure}
    \begin{tikzpicture}[nodes = {font=\sffamily}]
      \foreach \percent/\name in {
        52.1/PushEvent,
        10.9/CreateEvent,
        10.0/WatchEvent,
        9.1/IssueCommentEvent,
        5.0/IssuesEvent,
        4.3/PullRequestEvent,
        3.2/ForkEvent,
        5.4/Other events
%        1.7/DeleteEvent,
%        1.3/PullRequestReviewCommentEvent,
%        1.0/GollumEvent,
%        0.6/CommitCommentEvent,
%        0.4/ReleaseEvent,
%        0.3/MemberEvent,
%        0.1/PublicEvent
        } {
          \ifx\percent\empty\else               % If \percent is empty, do nothing
            \global\advance\cyclecount by 1     % Advance cyclecount
            \global\advance\ind by 1            % Advance list index
            \ifnum3<\cyclecount                 % If cyclecount is larger than list
              \global\cyclecount=0              %   reset cyclecount and
              \global\ind=0                     %   reset list index
            \fi
            \pgfmathparse{\cyclelist[\the\ind]} % Get color from cycle list
            \edef\color{\pgfmathresult}         %   and store as \color
            % Draw angle and set labels
            \draw[fill={\color!50},draw={\color}] (0,0) -- (\angle:\radius)
              arc (\angle:\angle+\percent*3.6:\radius) -- cycle;
            \node at (\angle+0.5*\percent*3.6:0.7*\radius) {\percent\,\%};
            \node[pin=\angle+0.5*\percent*3.6:\name]
              at (\angle+0.5*\percent*3.6:\radius) {};
            \pgfmathparse{\angle+\percent*3.6}  % Advance angle
            \xdef\angle{\pgfmathresult}         %   and store in \angle
          \fi
        };
    \end{tikzpicture}
    \end{figure}
\end{frame}


\end{document}
